From our NED position derived from raw GPS measurements and our EKF estimated position and velocity, we can analyze the accuracy of the measurements. GPS converted to NED provides us with absolute position, which is extremely useful over a long period of time for long-term accuracy, since it does not drift over time, it provides a consistent positioning based on the signals it receives from the satellite. Furthermore, since GPS has low long-term drift, it provides more accurate positioning. On the other hand, GPS is also prone to noise and can be delayed with a low update rate of only a few Hz which makes it less suitable for UAVs that require real-time high-frequency updates to the data. EKF estimated position and velocity fuse both IMU measurements and GPS measurements providing a filtered estimate with high update rates of a few 100 Hz to allow the EKF to update the position at a faster rate than GPS. Since the EKF accounts for sensor drift, it is able to adjust the position estimate accordingly and smooth out noisy GPS measurements. Our data shows that EKF provided a better estimate of the position and velocity of the UAV, with velocity being less noisy as seen from the graph and the position estimate being identical to what was provided from the NED positions obtained from GPS raw measurements. We were able to see that the UAV took off and landed twice, with the first flight being approximately 10 m in height, and the second flight being approximately 31 m in height. The accelerometer bias converged to $b_x$ converging to
0.164 $m/s^2$, $b_y$ converging to -0.252 $m/s^2$, and $b_z$ converging to -0.087 $m/s^2$. The report shows a successful implementation of an extended kalman filter to better estimate the position and velocity of the UAV in flight.