This paper will describe the implementation of coordinate transformation from GPS raw measurement (longitude, latitude, and altitude) to ECEF to NED (x – North, y – East, z – Down) positions with and without a data fusion algorithm – the extended Kalman filter (EKF). The focus of this report is leveraging the EKF to provide a better estimate of the NED positions (in m) and NED velocities (in m/s) by accounting for the difference in update rate between GPS and IMU through manipulating the sensor update frequencies. From the initial code provided and the default plots, it can be deduced that the UAV takes off and lands twice. The total simulation time of the UAV’s flight is 2083 seconds, with the first 32 seconds having irregular data. It takes off the first time at 113 seconds, and lands at 483 seconds. The UAV takes off a 2nd time at 1443 seconds and lands at 1685 seconds. As the report below will show, without implementing an extended Kalman filter (EKF), the estimate of the NED positions is not as accurate and the EKF provides a more accurate state estimate by considering the difference in GPS and IMU update rates and relying on frequent IMU updates for short-term dynamics and less frequent GPS updates for long-term position accuracy. The EKF balances these updates by adjusting the Kalman gain based on the measurement noise and update frequencies. The report will cover the different formulas used to convert from GPS coordinates to NED positions and to implement the extended Kalman filter. It will also provide physical meanings of the state variables and covariance used in EKF, and why specific initialization values were chosen for the states as well as the process and measurement noise covariance, with the respective code.